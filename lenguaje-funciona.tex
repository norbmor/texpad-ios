%\listfiles
\makeatletter\let\ifGm@compatii\relax\makeatother
\documentclass[compress]{beamer}
\usepackage[spanish]{babel}
\usepackage[utf8]{inputenc}
\usepackage{graphicx}
\usepackage{Alegreya} %% Option 'black' gives heavier bold face 
\usepackage[T1]{fontenc}
\usepackage{textcomp}
%\usepackage[altbullet]{lucidabr}   
\usepackage{adjustbox}
 \usepackage{pstricks}
  \usepackage{pst-jtree}
\usepackage{expex}
\usepackage{tipa}
\usepackage{float}
\usepackage{gb4e}

\usetheme{CambridgeUS01}

%\usecolortheme{beaver}


\definecolor{fond}{RGB}{240,240,240}
    \beamerdefaultoverlayspecification{<+->}
    \setbeamercovered{transparent=25}


	\title{¿Qué es el lenguaje? Una respuesta simple}
	\author{Norberto Moreno Quibén\\
       norberto.morenoquibe@uah.es}
	\date{}

    \begin{document}

\section{El lenguaje}
\subsection{Introducción}

\begin{frame}
  \titlepage
\end{frame}



 	\begin{frame}

 \frametitle{¿Qué es lenguaje?}

\begin{alertblock}<2->{Una respuesta simple desde la neurociencia cognitiva}
  \begin{itemize}
	  \item<3->%
Una facultad cognitiva individual de naturaleza computacional que permite al individuo relacionar sonido y significado
\item<4->%
Esta facultad produce o \alert{genera} representaciones mentales que representan las propiedades esenciales de un objeto mental.
\item<5->%
Esta representación es la causa del comportamiento verbal que observamos.
\item<6->La explicación del comportamiento en términos de representaciones mentales es la base de las distintas ciencias cognitivas.
  \end{itemize}
\end{alertblock}

 \end{frame}


	\begin{frame}
	  \frametitle{Una respuesta simple}
	\begin{alertblock}{Implicaciones para la psicología}
	\begin{itemize}
	\item<2->%
	El lenguaje es una facultad cognitiva y por tanto es una \alert{propiedad de la mente}
	\end{itemize}
	
	\end{alertblock}
	\end{frame}


	\begin{frame}
	  \frametitle{Una respuesta simple}
	\begin{alertblock}{Implicaciones para la biología}
		\begin{itemize}
		\item<2->%
		El lenguaje es una facultad cognitiva y por tanto es una propiedad de la mente, y en consecuencia una \alert{propiedad del cerebro humano} (en último término del genoma humano).
		\end{itemize}
		
	\end{alertblock}
	\end{frame}

	\begin{frame}
	  \frametitle{Una respuesta simple}
	\begin{alertblock}{Implicaciones para la antropología}
		\begin{itemize}
		\item<2->% 
		El lenguaje es un \alert{sistema computacional} que opera sobre símbolos y que genera representaciones.
		\end{itemize}
		
	\end{alertblock}
	\end{frame}
	\begin{frame}
	  \frametitle{Una respuesta simple}
	\begin{alertblock}{Implicaciones para la sociología}
		\begin{itemize}
		\item<2->% 
		El lenguaje es una facultad cognitiva \alert{individual} y por tanto personal.
		\end{itemize}
		
	\end{alertblock}
	\end{frame}
	\begin{frame}
	  \frametitle{Una respuesta simple}
	\begin{alertblock}{Implicaciones para la filosofía}
\begin{itemize}
\item<2->% 	
	El lenguaje es una facultad, una ``potencia'' de la mente que nos entronca con la visión \alert{racionalista} de la naturaleza humana, en oposición a  la visión empirista de la misma.
\end{itemize}

	\end{alertblock}
	\end{frame}
	
	\begin{frame}
		\frametitle{Otros objetos mentales producto de representaciones mentales}
		\begin{figure}[H]
		\begin{center}
		 \begin{pspicture}(0,-.5)(5,3)
		\pscircle*(.9,.5){.55}
		\pscircle*(3.6,.5){.55}
		\pscircle*(2.25,4.5){.55}
		\pspolygon[linestyle=none, fillstyle=solid,  fillcolor=white](.9,.5)(3.6,.5)(2.25,4.5)
		 \end{pspicture}
		\end{center}
		\caption{Triángulo construido por el sistema visual.}\label{ilusion}
		\end{figure}
	\end{frame}

	
	\begin{frame}
		\frametitle{Otros objetos mentales producto de representaciones mentales}
		\begin{figure}[H]
		\begin{center}
		  \includegraphics[width=2.3in]{images/13sided.eps}
		\end{center}
		\caption{Un polígono de 13 lados ilusorio: el triskaidecágono}\label{ilusiontrisk}
		\end{figure}
	\end{frame}	

\begin{frame}
	El sistema computacional del lenguaje es un sistema modular compuesto por cuatro subcomponentes y un depósito de palabras:
	\begin{itemize}
		\item \textbf{Lexicón}: el diccionario mental del hablante.
		\item \textbf{Fonética-Fonología}: construye la estructura de sonidos del lenguaje.
		\item \textbf{Morfología}: construye la estructura interna de las palabras.
		\item \textbf{Sintaxis}: construye la estructura de los grupos de palabras y de las oraciones del lenguaje.
		\item \textbf{Semántica}: construye el significado descriptivo de los grupos de palabras del lenguaje. 
	\end{itemize}
\end{frame}


\begin{frame}
\begin{adjustbox}{max size={\textwidth}{\textheight}}
	\begin{psmatrix}[colsep=1.8,rowsep=1.8]
&	& Sintaxis \\
&	& \psframebox[linearc=0.2,cornersize=absolute,framesep=6pt]{\tabular{c}Competencia\\ sintáctica\endtabular} \\
\tabular{c}Fonética\\Fonología\endtabular &  \psframebox[linearc=0.2,cornersize=absolute,framesep=6pt]{\tabular{c}Competencia\\ fonológica\endtabular} &  \psframebox[linearc=0.2,cornersize=absolute,framesep=20pt, fillstyle=solid,fillcolor=fond]{\textbf{Competencia}} & \psframebox[linearc=0.2,cornersize=absolute,framesep=6pt]{\tabular{c}Competencia\\ morfológica\endtabular} & Morfología \\
& &  \psframebox[linearc=0.2,cornersize=absolute,framesep=6pt]{\tabular{c}Competencia\\ semántica\endtabular} \\
& & Semántica
	\end{psmatrix}
\ncline{2,3}{3,3}
\ncline{3,2}{3,3}
\ncline{3,3}{3,4}
\ncline{3,3}{4,3}
\ncline[nodesepA=2pt]{->}{1,3}{2,3}
\ncline{->}{3,1}{3,2}
\ncline[nodesepA=2pt]{->}{5,3}{4,3}
\ncline[nodesepA=2pt]{->}{3,5}{3,4}
\end{adjustbox}
\end{frame}

	\section{Niveles de representación}
	\subsection{La oración}
	\begin{frame}
	\frametitle{La oración es la unidad de procesamiento}
	\begin{itemize}

	  \item Las oraciones constituyen la base del análisis de la lingüística formal.
	  \item Una oración es un objeto complejo que exige para su explicación un análisis en distintos componentes o representaciones.
	\end{itemize}
	\end{frame}
\subsection{La noción de nivel de representación}
	\begin{frame}
	\frametitle{La suma de los niveles de representación es una oración}
	\pex[*] El soldado llegó
	
\a Análisis fonológico: \textipa{[elsolda\dh oJeGo]}
\a Análisis silábico: $[_\sigma$\textipa{el}$]$$[_\sigma$\textipa{sol}$]$$[_\sigma$\textipa{da}$]$$[_\sigma$\textipa{Do}$]$$[_\sigma$\textipa{Je}$]$$[_\sigma$\textipa{Go}$]$
\a Análisis morfológico:\\ \textipa{el}[\textsc{det. masc. sing
	    3p}]\\
	    \textipa{solda\dh o}[\textsc{n masc. sg. 3p}]\\
	    \textipa{JeGo}[\textsc{v pres. sg. 3p}]
\xe
	\end{frame}


	\begin{frame}[fragile]
	\frametitle{Análisis sintáctico}
	\begin{center}
	\jtree[xunit=3.6em,yunit=2em]
	\def\\{[labelgapb=-1.2ex]}%
	%\everymath={\rm}%
	\! = {SFLEX}
	   :[scaleby=1.2]{SN}(<vartri>{El soldado}@A1 ) {FLEX$'$}
	   :({$\textrm {FLEX^0}$}\\{\rnode[b]{A2}{$\scriptstyle \textrm{[3,SG]}$}})
	   {SV}<vert>{V$'$}<vert>{llegó}.
	\ncbar[angleA=-90,angleB=-90,armA=1em,armB=1em,linearc=.6ex]{->}{A2}{A1}
	\endjtree

	\end{center}

	\end{frame}


	\begin{frame}

	\frametitle{Análisis semántico}
	\begin{center}
	La oración \emph{El soldado llegó} es verdadera si la entidad
	denotada por $[_{\textrm {SN}}$ el soldado] tiene la propiedad denotada
	por $[_{\textrm {SV}}$ llegó].
	\end{center}
	\end{frame}

\subsection{La suma de los niveles de representación}
	\begin{frame}
	\frametitle{La forma de la gramática}
	\[
	\begin{psmatrix}[colsep=1.5cm,rowsep=1.5cm]
	         &   \textrm{Léxico}               &  \\
	  \textrm{Derivación}  & &\textrm{Flexión} \\
	 & \textrm{Sintaxis} &\\
	 \textrm{Fonología}  &&\textrm{Semántica} 
	\psset{arrows=->,labelsep=3pt,nodesep=3pt}
	\ncline{1,2}{3,2}%
	\ncline{3,2}{4,1}%
	\ncline{3,2}{4,3}%
	\ncline{2,1}{1,2}%
	\ncline{2,3}{1,2}%
	\end{psmatrix}
	\]
	\end{frame}

	\begin{frame}
	\frametitle{La representación de la oración}

	El resultado es una representación final que incluye información
	fonológica, información morfológica, información sintáctica e
	información semántica.

	\begin{equation*}
	    \mathfrak{R}_{\textrm{El soldado llegó}} = \{\mathfrak{R}_{FON}, \mathfrak{R}_{MOR}, \mathfrak{R}_{SINT}, \mathfrak{R}_{SEM}\}
	\end{equation*}


	\end{frame}




	\section{Lenguaje y cerebro}

	\begin{frame}

	\frametitle{La realidad psicológica de la gramática}

	\begin{itemize}
	\item La forma de la gramática refleja la realidad de la organización de la Facultad de Lenguaje en la mente/cerebro.

	\item Sería posible encontrar correlatos neuronales/cognitivos para los componentes de la gramática.

	\end{itemize}
	\end{frame}


	\subsection{Disociación del lenguaje y de la cognición}
	\begin{frame}
		\frametitle{Síndrome de Williams, SW}
		\begin{itemize}
		\item SW: enfermedad de naturaleza genética con afectación del cromosoma 7.
		\item Bajo CI: alrededor de 57 (media 102).
		\item Dificultades de orientación espaciales. Hiperactividad lingüística con excepcional dominio del lenguaje.
		\item (Enumeración de animales) \alert{Brontosaurio, hipopótamo, iguanadón, ballena}\ldots (Edad: 15 años).
		\item ``Estás hablando con un escritor profesional. Mis libros tendrán drama, acción y aventura. Todo el mundo querrá leerlos.'' (Edad: 15 años)
		\end{itemize}
	\end{frame}
	
	\subsection{Análisis de errores}
	\begin{frame}
	\frametitle{Análisis de errores}

	\pex[*]
\a \ldots an indigent sex-\alert{offens}er (Elección de raíz derivativa: \alert{offend}er)
\a I can see a time in the no distan\alert{ce} future. (Sufijo derivativo: dista\alert{nt})
\a \ldots kill two \alert{stone}s with a \alert{bird} \ldots (Cambio de palabra: \alert{bird}s - \alert{stone})
\xe
	\end{frame}

	\begin{frame}
	\subsection{Afasia}
	\frametitle{Afasia de Broca: jerga agramatical}

	FS: afasia a causa de infarto cerebral
	\pex[*]
\a \ldots poi ancora spesso \alert{andare} [allo] studio. [In] via C. [a] il Pariolo
\a Ausencia de flexión verbal = \alert{andare} por \alert{vado} (1sg presente)
\a \ldots c'e \alert{la mia studia} ancora aperto.
\xe
	
\begin{itemize}
	  \item Elección incorrecta de la flexión nominal = morfemas
	  femeninos, l\alert{a} mi\alert{a} studi\alert{a}, por morfemas masculinos, \alert{e}l mi\alert{o} studi\alert{o}
	\end{itemize}
\end{frame}



	\begin{frame}
	\frametitle{Agramatismo}
	\begin{itemize}
	\item FS no tiene problemas a la hora de seleccionar el lexema
	correcto con el significado adecuado. No presenta una parafasia
	semántica.
	\item FS presenta rasgos de agramatismo: problemas a la hora de
	construir sintácticamente la oración.
	\end{itemize}

	\end{frame}


	\begin{frame}
	\frametitle{Distinción entre flexión y derivación} En tareas de
	repetición de palabras aisladas, FS produjo los siguientes datos

	\begin{itemize}
	\item 96.3 \% de los errores son de tipo flexivo y afectan a la
	flexión nominal o a la flexión verbal.

	\item 3.7 \% de errores en morfemas derivativos, en la elección de
	las raíces y los conceptos.
	\end{itemize}

	\end{frame}

	\begin{frame}
	\frametitle{Dificultad en la flexión}


	\begin{itemize}
	\item \emph{car}-\alert{o} (masc. sg.) por \emph{car}-\alert{i} (masc.
	pl.), \emph{car}-\alert{a} (fem. sg.), \emph{car}-\alert{e} (fem.
	pl.) (Un único morfema flexivo para todo el paradigma)
	\item \emph{fort}-\alert{e} (masc/fem. sg.) por \emph{fort}-\alert{i} (masc/fem. pl.).
	\end{itemize}

	\end{frame}

	\section{Conclusión}

	\begin{frame}
	\frametitle{Conclusión}
	\begin{itemize}
	\item Los desórdenes evolutivos del lenguajes y las afasias demuestran la disociación entre inteligencia y lenguaje, así como entre los distintos componentes de la facultad de lenguaje.  
	\item  La distinción entre flexión y derivación debe estar presente
	en la gramática mental de tal modo que una de ellas pueda verse
	afectada mientras que la otra permanezca intacta.
	  \item La distinción entre derivación y flexión es
	psicológicamente real. No es solo una distinción formal, puesto que
	tiene un correlato neurofuncional.
	\end{itemize}



	\end{frame}
    \end{document}
